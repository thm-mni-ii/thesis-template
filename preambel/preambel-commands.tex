%%% Packages for LaTeX - programming
%
% Define commands that don't eat spaces.
\usepackage{xspace}
% IfThenElse
\usepackage{ifthen}
%%% Doc: ftp://tug.ctan.org/pub/tex-archive/macros/latex/contrib/oberdiek/ifpdf.sty
% command for testing for pdf-creation
\usepackage{ifpdf} %\ifpdf \else \fi

%%% Internal Commands: ----------------------------------------------
\makeatletter
%
\providecommand{\IfPackageLoaded}[2]{\@ifpackageloaded{#1}{#2}{}}
\providecommand{\IfPackageNotLoaded}[2]{\@ifpackageloaded{#1}{}{#2}}
\providecommand{\IfElsePackageLoaded}[3]{\@ifpackageloaded{#1}{#2}{#3}}
%
\newboolean{chapteravailable}%
\setboolean{chapteravailable}{false}%

\ifcsname chapter\endcsname
  \setboolean{chapteravailable}{true}%
\else
  \setboolean{chapteravailable}{false}%
\fi


\providecommand{\IfChapterDefined}[1]{\ifthenelse{\boolean{chapteravailable}}{#1}{}}%
\providecommand{\IfElseChapterDefined}[2]{\ifthenelse{\boolean{chapteravailable}}{#1}{#2}}%

\providecommand{\IfDefined}[2]{%
\ifcsname #1\endcsname
   #2 %
\else
     % do nothing
\fi
}

\providecommand{\IfElseDefined}[3]{%
\ifcsname #1\endcsname
   #2 %
\else
   #3 %
\fi
}

\providecommand{\IfElseUnDefined}[3]{%
\ifcsname #1\endcsname
   #3 %
\else
   #2 %
\fi
}


%
% Check for 'draft' mode - commands.
\newcommand{\IfNotDraft}[1]{\ifx\@draft\@undefined #1 \fi}
\newcommand{\IfNotDraftElse}[2]{\ifx\@draft\@undefined #1 \else #2 \fi}
\newcommand{\IfDraft}[1]{\ifx\@draft\@undefined \else #1 \fi}
%

% Definde frontmatter, mainmatter and backmatter if not defined
\@ifundefined{frontmatter}{%
   \newcommand{\frontmatter}{%
      %In Roemischen Buchstaben nummerieren (i, ii, iii)
      \pagenumbering{roman}
   }
}{}
\@ifundefined{mainmatter}{%
   % scrpage2 benoetigt den folgenden switch
   % wenn \mainmatter definiert ist.
   \newif\if@mainmatter\@mainmattertrue
   \newcommand{\mainmatter}{%
      % -- Seitennummerierung auf Arabische Zahlen zuruecksetzen (1,2,3)
      \pagenumbering{arabic}%
      \setcounter{page}{1}%
   }
}{}
\@ifundefined{backmatter}{%
   \newcommand{\backmatter}{
      %In Roemischen Buchstaben nummerieren (i, ii, iii)
      \pagenumbering{roman}
   }
}{}

% Pakete speichern die spaeter geladen werden sollen
\newcommand{\LoadPackagesNow}{}
\newcommand{\LoadPackageLater}[1]{%
   \g@addto@macro{\LoadPackagesNow}{%
      \usepackage{#1}%
   }%
}

\newcommand*{\printGenerativeAIDeclaration}{%
\ifcsdef{declareUseOfGenerativeAITool}{%
\ifthenelse{\equal{\lang}{ngerman}}%
{%
\section*{Erklärung zur Verwendung von generativer KI}
In Übereinstimmung mit den Empfehlungen der
Deutschen Forschungsgemeinschaft (DFG)\footnote{DFG formuliert Richtlinien für den Umgang mit generativen Modellen für Text- und Bild: \url{https://www.dfg.de/en/news/news-topics/announcements-proposals/2023/info-wissenschaft-23-72}}
und denen der Zeitschrift Theoretical Computer Science\footnote{Erklärung zur Verwendung von generativer KI in wissenschaftlichen Arbeiten: \url{https://www.sciencedirect.com/journal/theoretical-computer-science/publish/guide-for-authors}}
erkläre ich (der Autor/die Autorin) hiermit den Einsatz von generativer KI.

Bei der Vorbereitung dieser Arbeit habe ich 
\declareUseOfGenerativeAITool\ verwendet, um ausschließlich die Lesbarkeit und Sprache zu verbessern.
Nach der Verwendung von \declareUseOfGenerativeAITool\ habe ich 
den Inhalt überprüft und nach Bedarf bearbeitet und übernehme die volle Verantwortung 
für den Inhalt dieser Arbeit.
}
{%
\section*{Declaration of the use of Generative AI}
In accordance with the recommendation of 
the German Research Foundation (DFG - Deutsche Forschungsgemeinschaft)\footnote{DFG Formulates Guidelines for Dealing with Generative Models for Text and Image Creation: \url{https://www.dfg.de/en/news/news-topics/announcements-proposals/2023/info-wissenschaft-23-72}}
and that of the journal Theoretical Computer Science\footnote{Declaration of generative AI in scientific writing: \url{https://www.sciencedirect.com/journal/theoretical-computer-science/publish/guide-for-authors}}
I (the author) herby declare the use of generative AI.

During the preparation of this work I used 
\declareUseOfGenerativeAITool\ in order to improve readability and language, only.
After using \declareUseOfGenerativeAITool, I reviewed and edited 
the content as needed and take full responsibility 
for the content of this thesis.
}
}{}
}

\newcommand*\printDeclarationOfIndependence{%
\ifthenelse{\equal{\lang}{ngerman}}%
{%
\section*{Erklärung der Selbstständigkeit}
\thispagestyle{empty}
Hiermit versichere ich, die vorliegende Arbeit selbstständig verfasst und keine anderen als die angegebenen Quellen und Hilfsmittel benutzt sowie die Zitate deutlich kenntlich gemacht zu haben.
\vspace{4\baselineskip}\\
Gießen, den \today \hfill \theAuthor
\vspace{4\baselineskip}\\
}
{%
\section*{Declaration of Independence}
\thispagestyle{empty}
I hereby declare that I have composed the present work independently and have not used any sources or aids other than those cited, and that all quotations have been clearly indicated.
\vspace{4\baselineskip}\\
Gießen, on \today \hfill \theAuthor
\vspace{4\baselineskip}\\
}
}

\let\oldauthor\author
\renewcommand{\author}[1]{\oldauthor{#1}\newcommand{\theAuthor}{#1}}

\let\oldtitle\title
\renewcommand{\title}[1]{\oldtitle{#1}\newcommand{\theTitle}{#1}}

\newcommand*{\academicTitle}[1]{\newcommand{\theAcedemicTitle}{#1}}
\newcommand*{\thesis}[1]{\newcommand{\theThesis}{#1}}
\newcommand*{\firstReferee}[1]{\newcommand{\theFirstReferee}{#1}}
\newcommand*{\secondReferee}[1]{\newcommand{\theSecondReferee}{#1}}

\makeatother
%%% ----------------------------------------------------------------