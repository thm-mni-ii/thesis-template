% Alles was man wissen muss um das Konzept zu verstehen.

\chapter{Hintergrund}

% Beispiel: Zwei Boxen nebeneinander für Vergleiche oder Gegenüberstellungen
\begin{figure}[htbp] % [H] verwenden, wenn es wirklich fix stehen soll
\vspace{0.4cm}
\begin{tcbraster}[raster columns=2, 
raster equal height, % optional feste Höhe
raster column skip=5mm]
\begin{tcolorbox}[title=\textbf{Box A}, colback=white, arc=3mm, boxrule=1.5pt]
Boxen eignen sich, um Inhalte übersichtlich hervorzuheben. Ideal für Listen oder zentrale Punkte.
\begin{itemize}
\item Punkt A
\item Punkt B
\item Punkt C
\end{itemize}
\end{tcolorbox}
\begin{tcolorbox}[title=\textbf{Box B}, colback=white, arc=3mm, boxrule=1.5pt]
Diese Box kann als Gegenstück oder für Vergleiche genutzt werden, z.\,B. Unterschiede oder Alternativen.
\begin{itemize}
\item Punkt A
\item Punkt B
\item Punkt C
\end{itemize}
\end{tcolorbox}
\end{tcbraster}
\end{figure}
