\chapter{Konzept}

% Beispiel: Bild + Infobox nebeneinander
\begin{figure}[htbp] % [H] verwenden, wenn es wirklich fix stehen soll
\begin{tcbraster}[raster columns=2, raster height=13.5cm, raster column skip=5mm]
\begin{tcolorbox}[colback=white, colframe=white, boxrule=0pt]
\centering
\fbox{\includegraphics[width=\textwidth, height=6cm, keepaspectratio]{images/Logo_THM_MNI}}
\\ \vspace{0.2cm}
\textit{Bildunterschrift}
\end{tcolorbox}
\begin{tcolorbox}[colback=white, arc=3mm, boxrule=1.5pt]
\textbf{Bild + Infobox} \\
\\
Diese Art der Darstellung eignet sich besonders gut, um kurze Inhalte übersichtlich und visuell begleitet zu präsentieren.
\begin{itemize}[leftmargin=*, itemsep=4pt]
\item Visualisierung
\item Erläuterung
\item Auflistung
\item Begleittext
\end{itemize}
\end{tcolorbox}
\end{tcbraster}
\end{figure}