\chapter{Einführung}
\begin{quote}
Lorem ipsum dolor sit amet, consectetur adipisicing elit, sed do eiusmod tempor incididunt ut labore et dolore magna aliqua. Ut enim ad minim veniam \cite{smith2020example} , quis nostrud exercitation ullamco laboris nisi ut aliquip ex ea commodo consequat. Duis aute irure dolor in reprehenderit in voluptate velit \cite{doe2019research}  esse cillum dolore eu fugiat nulla pariatur. Excepteur sint occaecat cupidatat non proident, sunt in culpa qui officia deserunt mollit anim id est laborum \cite{lee2018conference} 
\end{quote} 

\section{Problembeschreibung/Motivation}

Et harum quidem rerum facilis est et expedita distinctio. Nam libero tempore, cum soluta nobis est eligendi optio cumque nihil impedit quo minus id quod maxime placeat facere possimus, omnis voluptas assumenda est, omnis dolor repellendus.

Sed ut perspiciatis unde omnis iste natus error sit voluptatem accusantium doloremque laudantium, totam rem aperiam, eaque ipsa quae ab illo inventore \cite{miller2017chapter}  veritatis et quasi architecto beatae vitae dicta sunt explicabo. Nemo enim ipsam voluptatem quia voluptas sit aspernatur aut \cite{website2021example}  odit aut fugit, sed quia consequuntur magni dolores eos qui ratione voluptatem sequi nesciunt.

% Beispiel: Hinweisbox (kurze Tipps oder Zusammenfassungen)
\begin{tcolorbox}[arc=3mm, boxrule=0pt, leftrule=4pt, left=10pt]
Diese Art von Box eignet sich besonders gut, um zwischen sachlichen Abschnitten kurze Zusammenfassungen oder wichtige Hinweise einzufügen – etwa am Kapitelende oder zwischen methodischen Schritten.
\end{tcolorbox}

\section{Ziele dieser Arbeit}

% Beispiel: Zentrale Infobox (Forschungsfragen, Definitionen, wichtige Aspekte)
\begin{center}
\begin{minipage}{\textwidth}
\begin{tcolorbox}[title=\textbf{Forschungsfragen, Infobox, Definition etc.}, left=1mm, right=1mm]
\raggedright
Diese Box dient der übersichtlichen Darstellung wichtiger Inhalte wie Forschungsfragen, Definitionen oder methodischer Hinweise. Sie kann genutzt werden, um zentrale Aspekte optisch hervorzuheben und den Lesefluss zu unterstützen. \\
\end{tcolorbox}
\end{minipage}
\end{center}

\section{Vorgehensweise/Methode}

\section{Abgrenzung}

\section{Struktur der Arbeit}
